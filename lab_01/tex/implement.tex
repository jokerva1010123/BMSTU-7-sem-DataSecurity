\chapter{Технологическая часть}

В данном разделе будут рассмотрены средства реализации, а также представлены листинги реализаций алгоритма шифрования машины <<Энигма>>.

\section{Средства реализации}
В данной работе для реализации был выбран язык программирования $C$. Данный язык удоволетворяет поставленным критериям по средствам реализации.

\section{Реализация алгоритма}

В листингах \ref{lst:enigma1} представлена реализация алгоритма шифрования машины <<Энигма>>.

\begin{center}
    \captionsetup{justification=raggedright,singlelinecheck=off}
    \begin{lstlisting}[label=lst:enigma1,caption=Реализация алгоритма шифрования машины <<Энигма>>]
int main(int argc, char* argv[]) { 
	int stringSize;
	char *string;

	FILE *f = fopen("input.txt", "rb");
	if (!f) {
		printf("Can not open input file.");
		return 1;
	}

	if (fscanf(f, "%d\n", &stringSize) != 1) {
		printf("Can not input size of symbols.");
		return 1;
	}
	
	string = malloc(stringSize); 
	for (int i = 0; i < stringSize; i++) {
		if (fscanf(f, "%c", &string[i]) != 1) {
			printf("Can not read symbols.");
			return 1;
		}
	}
	
	int scrollCounter = 0;
	for (int i = 0; i < stringSize; i++) {
		if (string[i] != ' ') {
			string[i] = secondSwitch(
				secondFastRotorEncoding(
				secondMiddleRotorEncoding(
				secondSlowRotorEncoding(
				getReflectedChar(
				firstSlowRotorEncoding(
				firstMiddleRotorEncoding(
				firstFastRotorEncoding(
				firstSwitch(string[i])
				)))))))); 
			scrollCounter++;
			scrollFastRotor();
			if (scrollCounter % 26 == 0)
				scrollMiddleRotor();
			if (scrollCounter % (26 * 26) == 0)
				scrollSlowRotor();
		}
	}

	for (int i = 0; i < stringSize; i++) printf("%c", string[i]);
	printf("\n");
}
\end{lstlisting}
\end{center}
\section*{Вывод}

Были представлены листинги реализаций алгоритма шифрования в машине <<Энигма>> согласно алгоритму, представленному в первой части, а также проведено тестирование разработанной программы.