\chapter{Конструкторская часть}
В этом разделе будут представлены описания используемых типов данных, а также требования к программе.

\section{Описание используемых типов данных}

При реализации алгоритмов будут использованы следующие типы данных для соответствующих значений:
\begin{itemize}[label=---]
	\item набор роторов  - одномерный список чисел;
	\item рефлектор - одномерный список чисел;
	\item сообщение - список символов;
\end{itemize}

\section{Требования к программе}
Входными данными программы должен быть файл input.txt, содержащий количество байтов и символьных байтов, которые необходимо зашифровать или дешифровать.

Выходными данными программы является строка символов —-- результат зашифрования или дешифрования входной строки после использования машины «Энигма».

