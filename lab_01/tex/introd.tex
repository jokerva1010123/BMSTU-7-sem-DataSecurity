\chapter*{Введение}
\addcontentsline{toc}{chapter}{Введение}

Шифрование информации --- занятие, которым человек занимался ещё до начала первого тысячелетия, занятие, позволяющее защитить информацию от посторонних лиц. 
Существует большое число шифровальных алгоритмов, таких, как:
\begin{itemize}[label=---]
    \item шифр Цезаря;
    \item шифр Вернама;
    \item шифр Виженёра.
\end{itemize}


Шифровальная машина <<Энигма>> --- одна из самых известрых шифровальнных машин, использовавшихся для шифрования и расшифровывания секретных сообщений.

\textbf{Целью данной работы} является реализация в виде программы на языке программирования C или C++ аналога шифровальной машины <<Энигма>>, обеспечеие шифрования и расшифровки файла. 

Для достижения поставленной цели необходимо выполнить следующие задачи:
\begin{enumerate}[label=\arabic*)]
	\item изучить алгоритм работы шифровальной машины <<Энигма>>;
    \item реализовать алгоритм работы шифровальной машины <<Энигма>> в виде программы, обеспечив возможности шифрования и расшифровки текстового файла;
	\item протестировать разработанную программу, показать, что удаётся дешфировать все сообщения и получить исходные;
	\item описать и обосновать полученные результаты в отчёте о выполненной лабораторной работе, выполненном как расчётно-пояснительная записка к работе, содержащая три раздела: аналитический, конструкторский и технологический.
\end{enumerate}