\chapter{Конструкторская часть}
В этом разделе будут представлены описания используемых типов данных, а также схема алгоритма разрабатываемой программы.

\section{Описание используемых типов данных}

При реализации алгоритмов будут использованы следующие типы данных для соответствующих значений:
\begin{itemize}[label=---]
	\item матрица -- двумерный список символов;
	\item набор роторов  - матрица;
	\item набор рефлекторов - матрица;
	\item сообщение - список символов;
\end{itemize}


\section{Сведения о модулях программы}
Программа состоит из четырёх модулей:
\begin{enumerate}[label=\arabic*)]
	\item $main.c$ --- файл, содержащий точку входа;
    \item $menu.c$ --- файл, содержащий код меню программы;
    \item $rotors.c$ --- файл, содержайший значения различных роторов и рефлекторов;
    \item $enigma.c$ --- файл, содержащий алгоритм шифрации.
\end{enumerate}

\section{Разработка алгоритмов}
На рисунке \ref{img:enigma} представлена схема работы программы, реализующей шифровальную машину <<Энигма>>.

\imgScale{0.38}{enigma}{Схема работы программы, реализующей шифровальную машину <<Энигма>>}

Алгоритм шифрования прелагается реализовать согласно описанному в аналитическом разделе алгоритму.

\section*{Вывод}

В данном разделе были представлены описания используемых типов данных, а также схема алгоритма разрабатываемой программы.
